% Template for Cogsci submission with R Markdown

% Stuff changed from original Markdown PLOS Template
\documentclass[10pt, letterpaper]{article}

\usepackage{cogsci}
\usepackage{pslatex}
\usepackage{float}
\usepackage{caption}

% amsmath package, useful for mathematical formulas
\usepackage{amsmath}

% amssymb package, useful for mathematical symbols
\usepackage{amssymb}

% hyperref package, useful for hyperlinks
\usepackage{hyperref}

% graphicx package, useful for including eps and pdf graphics
% include graphics with the command \includegraphics
\usepackage{graphicx}

% Sweave(-like)
\usepackage{fancyvrb}
\DefineVerbatimEnvironment{Sinput}{Verbatim}{fontshape=sl}
\DefineVerbatimEnvironment{Soutput}{Verbatim}{}
\DefineVerbatimEnvironment{Scode}{Verbatim}{fontshape=sl}
\newenvironment{Schunk}{}{}
\DefineVerbatimEnvironment{Code}{Verbatim}{}
\DefineVerbatimEnvironment{CodeInput}{Verbatim}{fontshape=sl}
\DefineVerbatimEnvironment{CodeOutput}{Verbatim}{}
\newenvironment{CodeChunk}{}{}

% cite package, to clean up citations in the main text. Do not remove.
\usepackage{cite}

\usepackage{color}

% Use doublespacing - comment out for single spacing
%\usepackage{setspace}
%\doublespacing


% % Text layout
% \topmargin 0.0cm
% \oddsidemargin 0.5cm
% \evensidemargin 0.5cm
% \textwidth 16cm
% \textheight 21cm

\title{Children's understanding of simple polite markers}


\author{{\large \bf } \\ \texttt{} \\  \\}

\begin{document}

\maketitle

\begin{abstract}


\textbf{Keywords:}
Add your choice of indexing terms or keywords; kindly use a semi-colon;
between each term.
\end{abstract}

\section{Introduction}\label{introduction}

As adults, we use polite speech all the time.

Children produce polite speech early on.

Less work has looked at children's comprehension of polite speech.

For example, Nippold, Leonard, \& Anastopoulos (1982) looked at\ldots{}

In this current work, we seek to test what 2- to 4-year-old children
understand about polite speech. Specifically, we ask: (1) whether
children can reason about which speaker is being more ``polite'' or
``rude'' (or ``nice'' or ``mean''); (2) whether they understand possible
social consequences of being polite or impolite; and (3) how this
reasoning may change across development.

\section{Experiment 1}\label{experiment-1}

\subsection{Methods}\label{methods}

\subsubsection{Participants}\label{participants}

FIXME 3- and 4-year-olds (FIXME female, FIXME) were recruited from a
local preschool. An additional 3 children were tested but excluded due
to failure on the practice questions (2) or completion of fewer than
half of the test trials (1).

\subsubsection{Stimuli}\label{stimuli}

\subsubsection{Procedure}\label{procedure}

\subsection{Results and Discussion}\label{results-and-discussion}

\section{Experiment 2}\label{experiment-2}

\subsection{Methods}\label{methods-1}

\subsubsection{Participants}\label{participants-1}

\subsubsection{Stimuli}\label{stimuli-1}

\subsubsection{Procedure}\label{procedure-1}

\subsection{Results and Discussion}\label{results-and-discussion-1}

\section{Experiment 3}\label{experiment-3}

\subsection{Methods}\label{methods-2}

\subsubsection{Participants}\label{participants-2}

\subsubsection{Stimuli}\label{stimuli-2}

\subsubsection{Procedure}\label{procedure-2}

\subsection{Results and Discussion}\label{results-and-discussion-2}

\section{General Discussion}\label{general-discussion}

\section{References}\label{references}

\setlength{\parindent}{-0.1in} \setlength{\leftskip}{0.125in} \noindent

\hypertarget{refs}{}
\hypertarget{ref-nippold1982}{}
Nippold, M. A., Leonard, L. B., \& Anastopoulos, A. (1982). Development
in the use and understanding of polite forms in children. \emph{Journal
of Speech, Language, and Hearing Research}, \emph{25}(2), 193--202.

\end{document}
